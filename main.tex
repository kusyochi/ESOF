\documentclass[12pt,a4paper,article]{report}

\usepackage[brazil]{babel}
\usepackage[utf8]{inputenc}
\usepackage[brazil]{minitoc}
\usepackage{natbib}
\setcounter{minitocdepth}{1}
\author{Roger Tiago }
\date{Abril 2016}
\title{PDF e Word em Python}


\begin{document}




\maketitle



\newpage

\dominitoc
\tableofcontents

\newpage

\section{Introdução}
     Este documento visa auxiliar o trabalho com arquivos PDF e .docx utilizando Python.
    
\section{Documentos}
     Documentos em PDF e .docx são arquivos binários, oque dificulta o trabalho com esses arquivos, porem o python conta com módulos para facilitar estes processos.
\chapter{PDF} \label{Trabalhando com PDF}
\section{Documentos em PDF}
     Documentos em PDF são arquivos .pdf, PDF's podem conter textos, imagens e vídeos, aqui será focado em editar e ler textos de aquivos PDF, para isso usaremos o módulo PyPDF2.
    
\subsection{Extraindo textos de PDF's}
     O módulo PyPDF2 não consegue extrair imagens, gráficos e outras mídias dos arquivos, porém pode extrair textos e retornar como uma string.\par
     Para extrair um texto de um PDF é necessário a importação do módulo PyPDF2, cria-se um objeto para ler o pdf em modo binário e armazenar, é possivel obter o número de páginas do documento que começam a ser contadas a partir do 0, para extrair o texto propriamente dito usamo a função .getPage(n) no objeto criado anteriormente onde n representa a página que desejamos o texto(contada a partir do 0), após isso usa-se a função .extractText() no objeto que contém o pdf já converito em binário e o indice da pagina, será mostrada uma string contendo o texto. 
     
\subsubsection{Código exemplo}
    \begin{verbatim}
    >>> import PyPDF2 
    >>> pdfFileObj = open('meetingminutes.pdf', 'rb') 
    >>> pdfReader = PyPDF2.PdfFileReader(pdfFileObj)
    >>> pdfReader.numPages 
    >>> pageObj = pdfReader.getPage(0)
    >>> pageObj.extractText()
    \end{verbatim}
     
\subsection{Descriptografando PDF's}
    Alguns PDF's contam com um sistema de criptografia, e nao será possível ler o documento até que seja descriptografado. Todo objeto PdfFileReader conta com um atributo isEncrypted que se tiver o valor True nao será possível ler o PDF. Neste caso para ler um PDF criptogradfado basta usar a função .decrypt() no objeto com a senha do documento como argumento em string, se a senha estiver correta sera possivel acessar o documento, caso contrario ainda haverá erros.A função .decrypt() funciona apenas para o objeto e nao para o arquivo em si.
    
\subsubsection{Código exemplo}
    \begin{verbatim}
        >>> import PyPDF2 
        >>> pdfReader = PyPDF2.PdfFileReader(open('encrypted.pdf','rb'))
        >>> pdfReader.isEncrypted 
        >>> pdfReader.getPage(0)
    \end{verbatim}
    
\subsection{Criando PDF's}
    PyPDF2 pode criar novos PDF's, no entanto não é possível escrever arbitrariamente, apenas copiar páginas de outros PDF's, rotacionar, sobrepor e criptografar.
    PyPDF2 não pode editar diretamente um PDF, é necessário criar um novo PDF e importar textos já existentes.
    Para fazer isso é necessário abrir um ou mais PDF's  fontes como objetos com PdfFileReader, criar um novo objeto PdfFileWriter, copiar as páginas do objeto PdfFileReader para o objeto PdfFileWriter, finalmente usamo o objeto PdfFileWriter para escrever o PDF final.
    O objeto PdfFileWriter não criar o arquivo em si, para isto é necessário usar o método write().
    
    
\subsection{Copiando páginas}
    É possível utilizar o PyPDF2 para copiar páginas de um documento PDF para outro, isto permite combinar múltiplos arquivos PDF's, cortar páginas indesejadas e reordenas as páginas
    
\subsubsection{Código exemplo}
    \begin{verbatim}
        >>> import PyPDF2 
        >>> pdf1File = open('meetingminutes.pdf', 'rb') 
        >>> pdf2File = open('meetingminutes2.pdf', 'rb') 
        >>> pdf1Reader = PyPDF2.PdfFileReader(pdf1File) 
        >>> pdf2Reader = PyPDF2.PdfFileReader(pdf2File) 
        >>> pdfWriter = PyPDF2.PdfFileWriter()
        >>> for pageNum in range(pdf1Reader.numPages):
                 pageObj = pdf1Reader.getPage(pageNum)
                 pdfWriter.addPage(pageObj)
        >>> for pageNum in range(pdf2Reader.numPages):
                 pageObj = pdf2Reader.getPage(pageNum) 
                 pdfWriter.addPage(pageObj)
        >>> pdfOutputFile = open('combinedminutes.pdf', 'wb') 
        >>> pdfWriter.write(pdfOutputFile) 
        >>> pdfOutputFile.close() 
        >>> pdf1File.close() 
        >>> pdf2File.close()
    \end{verbatim}
    
\subsection{Rotacionando páginas}
    É possível rotacionar páginas de um PDF em incrementos de 90º com os métodos rotateClockWise() e rotateCounterClockWise() passando o argumento 90, 180 ou 270.
    
\subsubsection{Código exemplo}
    \begin{verbatim}
        >>> import PyPDF2 
        >>> minutesFile = open('meetingminutes.pdf', 'rb') 
        >>> pdfReader = PyPDF2.PdfFileReader(minutesFile) 
        >>> page = pdfReader.getPage(0) 
        >>> page.rotateClockwise(90) 
        >>> pdfWriter = PyPDF2.PdfFileWriter() 
        >>> pdfWriter.addPage(page) 
        >>> resultPdfFile = open('rotatedPage.pdf', 'wb') 
        >>> pdfWriter.write(resultPdfFile) 
        >>> resultPdfFile.close() 
        >>> minutesFile.close()
    \end{verbatim}
    
    
\subsection{Sobrepondo páginas}
    Utilizando PyPDF2 é possível sobrepor páginas, facilmente adioconando logos ou marcas d'água a multiplas páginas.

\subsubsection{Código exemplo}
    \begin{verbatim}
        >>> import PyPDF2 
        >>> minutesFile = open('meetingminutes.pdf', 'rb') 
        >>> pdfReader = PyPDF2.PdfFileReader(minutesFile) 
        >>> minutesFirstPage = pdfReader.getPage(0) 
        >>> pdfWatermarkReaderPyPDF2.PdfFileReader(open('watermark.pdf', 'rb')) 
        >>> minutesFirstPage.mergePage(pdfWatermarkReader.getPage(0)) 
        >>> pdfWriter = PyPDF2.PdfFileWriter() 
        >>> pdfWriter.addPage(minutesFirstPage)

        >>> for pageNum in range(1, pdfReader.numPages):
                pageObj = pdfReader.getPage(pageNum)
                pdfWriter.addPage(pageObj)
        >>> resultPdfFile = open('watermarkedCover.pdf', 'wb')
        >>> pdfWriter.write(resultPdfFile) 
        >>> minutesFile.close() 
        >>> resultPdfFile.close()
    \end{verbatim}
    
\subsection{Criptografando}
    O objeto PdfFileWriter pode adicionar uma senha a um arquivo PDF com o método .encrypt() passando como argumento a senha desejada.

\subsubsection{Código exemplo}
    \begin{verbatim}
        >>> import PyPDF2 
        >>> pdfFile = open('meetingminutes.pdf', 'rb') 
        >>> pdfReader = PyPDF2.PdfFileReader(pdfFile) 
        >>> pdfWriter = PyPDF2.PdfFileWriter() 
        >>> for pageNum in range(pdfReader.numPages):
                pdfWriter.addPage(pdfReader.getPage(pageNum))

        >>> pdfWriter.encrypt('swordfish') 
        >>> resultPdf = open('encryptedminutes.pdf', 'wb') 
        >>> pdfWriter.write(resultPdf) 
        >>> resultPdf.close()
    \end{verbatim}
    
\chapter{Word} \label{Trabalhando com documentos .docx}
\section{Documentos em .docx}
    Python pode criar e modificar documentos .docx, para isso é usado o módulo python-docx. Documentos .docx possuem muita estrutura, as quais são representadas por 3 diferentes tipos de data no módulo Python-Docx. No nível mais alto o objeto Document contém o documento inteiro e uma lista do objeto Paragraph para os parágrafos no documento(um paragrafo novo é criado a cada 'enter' teclado), cada um desses objetos Paragraph contém um ou mais objetos Run.
    Um texto em .docx é mais que uma string, ele contém fonte, cor e tamanho, um objeto Run é necessário sempre que o texto muda de estilo;
    
\subsection{Lendo Documentos em .docx}
    Para lermos um documento word usamo o módulo docx, chamando a função docx.Document() e passando o nome do arquivo como argumento obtemos o objeto Document, que contém o atributo paragraph que por sua vez é uma lista de objetos Paragraph. Usando a função len(doc.paragraphs) obtemos o número de parágrafos no documento. Cada objeto Paragraph contém uma string que representa o texto neste parágrafo. Cara objeto Paragraph contém um atributo Run que é uma lista de objetos Run estes objetos contém um atributo com o texto nessa Run em particular. Uma Run muda sempre que apertamos 'Enter' ou 'Tab'.
    
\subsubsection{Código Exemplo}
    \begin{verbatim}
        >>> import docx 
        >>> doc = docx.Document
        >>> len(doc.paragraphs)
        >>> doc.paragraphs[0].text 
        >>> doc.paragraphs[1].text
        >>> len(doc.paragraphs[1].runs)
        >>> doc.paragraphs[1].runs[0].text 
        >>> doc.paragraphs[1].runs[1].text
        >>> doc.paragraphs[1].runs[2].text
        >>> doc.paragraphs[1].runs[3].text
    \end{verbatim}
    
\subsection{Textos inteiros de documentos .docx}
    Se precisarmos apenas do texto em um documento .docx sem nos importarmos com as configurações de estilização do texto, podemos usar a função getText() passando o nome do arquivo como argumento. A função abre o documento, passa por todos os objetos Paragraph na lista paragraphs e acrescenta o texto na lista fullText.
    
\subsubsection{Código Exemplo}
    \begin{verbatim}
        import docx
        def getText(filename):
            doc = docx.Document(filename)
            fullText = []
            for para in doc.paragraphs:
                fullText.append(para.text)
            return '\n'.join(fullText)
    \end{verbatim}
    
\subsection{Escrevendo documentos em .docx}
    Usando docx.Document criamos um objeto de documento word vazio, com o método add\_paragraph() é possível adicionar um parágrafo ao documento e retorna uma referência ao parágrafo que foi adicionado, ao terminar para salvar o arquivo usamos o método save() passando como argumento o nome do arquivo desejado. É possível adicionar outros parágrafos com o método add\_paragraph() ou ainda adicionar textos ao final de um parágrafo existente com o método add\_run(), ambos os métodos aceitam um segundo argumento de estilização de texto.

\subsubsection{Código Exemplo}
    \begin{verbatim}
        >>> import docx
        >>> doc = docx.Document() 
        >>> doc.add_paragraph('Hello world!')
        <docx.text.Paragraph object at 0x0000000003B56F60>
        >>> paraObj1 = doc.add\_paragraph('This is a second paragraph.') 
        >>> paraObj2 = doc.add\_paragraph('This is a yet another paragraph.')
        >>> paraObj1.add_run(' This text is being added to 
        the second paragraph.') 
        <docx.text.Run object at 0x0000000003A2C860>
        >>> doc.add_paragraph('Hello world!', 'Title')
        >>> doc.save('multipleParagraphs.docx')
    \end{verbatim}

\subsection{Adicionando cabeçalhos}
    A função add\_heading() adiciona um parágrafo com um dos estilos de cabeçalho, são passados 2 argumentos para a função, sendo eles o texto que ficará no cabeçalho e um inteiro de 0 a 4, usando o argumento 0 na função o cabeçalho ficará no estilo título, os demais serão sub-cabeçalhos.
    
\subsubsection{Código Exemplo}
    \begin{verbatim}
        >>> doc = docx.Document() 
        >>> doc.add_heading('Header 0', 0) 
        <docx.text.Paragraph object at 0x00000000036CB3C8>
        >>> doc.save('headings.docx')
    \end{verbatim}
    
\subsection{Adicionando quebras de linhas e de páginas}
    Para adicionar uma quebra de linha ao invés de começar um novo parágrafo usa-se o método add\_break() no objeto Run anterior à quebra de linha, para adicionar uma quebra de página basta passar o argumento docx.text.WD\_BREAK.PAGE
    
\subsubsection{Código Exemplo}
    \begin{verbatim}
        >>> doc = docx.Document()
        >>> doc.add_paragraph('This is on the first page!') 
        <docx.text.Paragraph object at 0x0000000003785518> 
        >>> doc.paragraphs[0].runs[0].add_break(docx.text.WD_BREAK.PAGE)
        >>> doc.add_paragraph('This is on the second page!') 
        <docx.text.Paragraph object at 0x00000000037855F8> 
        >>> doc.save('twoPage.docx')
    \end{verbatim}
    
\subsection{Adicionando Imagens}
    O objeto Document conta o método add\_picture() que é capaz de adicionar uma imagem ao fim do documento, é possível utilizar a imagem com o tamanho padrão passando como argumento ao método apenas  o nome do arquivo de imagem, também é possível especificar a altura e largura da imagem usando as funções width=docx.shared.Inches(n) e
    height=docx.shared.Cm(n) como argumentos do método.
    
\subsubsection{Código Exemplo}
    \begin{verbatim}
        >>> doc.add_picture('zophie.png', width=docx.shared.Inches(1),
        height=docx.shared.Cm(4)) 
        <docx.shape.InlineShape object at 0x00000000036C7D30>
    \end{verbatim}
    
\chapter{Referências} \label{Bibliografia}

\subsection{Bibliografia}
    Trabalho realizado com base no livro \citet{sweigart2015automate}


\bibliographystyle{apa}
\bibliography{referencias}
\end{document}